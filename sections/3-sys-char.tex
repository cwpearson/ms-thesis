\chapter{System Characterization}
\label{ch:sys-char}

High-performance data movement in heterogeneous systems requires information about the properties of the communication links between system storage and compute components.
Although specifications of system components are often available\todo{cite some examples}, the real-world properties of these links depends on how applications use the links, and whether or not the links are shared between system components.
\todo{For example, Figure~\ref{fig:actual-perf} shows modeled and achieved cuda memcpy bandwidth.}
With full knowledge of link properties it is possible to derive an accurate model of link performance, that approach has two key barriers
\begin{itemize}
    \item Detailed link hardware properties are not available, e.g., when the link provides a competitive advantage for an OEM.
    \item Detailed link software properies are not available, e.g., when the drivers are proprietary.
    \item Even if a link is pysically present on the system, it may not be available to the application {e.g., due to bugs in the system configuration}
\end{itemize}
Instead of deriving a model of link performance from the ``first principles'' of link properties, this work attempts to generate an empirical model of performance of data movement in the system.
Section~\ref{sec:system-model} describes an overview of the system model.
Section~\ref{sec:topology-exploration} describes a method for discovering data sources, sinks, and communication paths in a system.
Section~\ref{sec:link-char} describes the approach to characterize communication links.

\begin{figure}[h]
    \centering
    \caption[\todo{short}]{\todo{long}}
    \centering
      \includegraphics[width=0.5\textwidth,draft]{../figures/placeholder.png}
    \label{fig:actual-perf}
\end{figure}

\section{System Model}
\label{sec:system-model}

The harware system is represented by a graph $G = \{E,V\}$ where $E$ is a set of edges representing communication links, and $V$ is a set of vertices representing data sources/sinks.
Associated with each edge is a performance model function $M: C,U \rightarrow P$ that maps a communication pattern $C$ and a link utilization $U$ to an achievable performance $P$.

The communication pattern $P$ has the following parameters:
\begin{itemize}
    \item The communication API or method used (e.g., \texttt{fread()}, CUDA unified memory page transfer).
    \item The number, size, and priority of pending transfers on the link.
\end{itemize}

The link utilization $U$ a set of extant communication patterns already sharing the link, separate from the communication of interest $C$.

This work considers PCIe~\todo{cite PCIe}, NVLink\cite{nvidia2017nvlink}, SATA~\todo{cite SATA}, and SMP bus links between CPU sockets.

Each vertex in $V$ represents a data source/sink. These represent data storage and computation components of the system.

This work considers CPU sockets, GPUs, and Linux block storage devices.

\section{Topology Exploration}
\label{sec:topology-exploration}

Using hwloc~\cite{broquedis2010hwloc} and NVML~\cite{nvidia2017nvml}.

\begin{figure}[h]
    \centering
    \caption[\todo{short}]{\todo{long}}
    \centering
      \includegraphics[width=0.5\textwidth,draft]{../figures/explore-topo-minsky.pdf}
    \label{fig:actual-perf}
\end{figure}

\begin{figure}[h]
    \centering
    \caption[\todo{short}]{\todo{long}}
    \centering
      \includegraphics[width=0.5\textwidth,draft]{../figures/explore-topo-dgx1.pdf}
    \label{fig:actual-perf}
\end{figure}


\section{Link Characterization}
\label{sec:link-char}

In general, communication link performance 

