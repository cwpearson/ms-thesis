\chapter{Explicit Memory Performance}
\label{ch:explicit}

\todo{combine summary tables into 1 big table}

This chapter is broken into three sections, describing explicit CPU-GPU transfers (Section~\ref{sec:explicit-cpu-gpu}) and GPU-GPU transfers (Section~\ref{sec:explicit-gpu-gpu}), both using \texttt{cudaMemcpy}.
It highlights cases where the observed logical communication performance deviates significantly from the symmetries present in the CUDA API, numactl API, and hardware.
Those deviances take the form of different performance on identical links, anisotropic link performance, or performance affected by device affinity.
Full results can be found in Appendix~\ref{ch:data}.

\section{CPU / CPU Transfers}
\label{sec:explicit-cpu-cpu}

Although not an operation used in CUDA applications, this section begins by examining CPU-CPU transfer performance by using multiple threads to read remote data in an attempt to saturate the SMP bus and memory controllers.
This attempts to simulate the maximum possible performance that the CUDA API could achieve while sending data between sockets for CPU-to-GPU transfers.
Algorithm~\ref{alg:explicit-cpu-cpu} describes the approach.


\begin{figure}[ht]
    \centering
    \begin{subfigure}[b]{0.3\textwidth}
        \includegraphics[width=\textwidth]{figures/generated/s822lc_cpu0-cpu1-dst.pdf}
        \caption{}
        \label{fig:s822lc-cpu0-cpu1}
    \end{subfigure}
    ~
    \begin{subfigure}[b]{0.3\textwidth}
        \includegraphics[width=\textwidth]{figures/generated/s822lc_cpu0-cpu1-src.pdf}
        \caption{}
        \label{fig:s822lc-cpu0-cpu1}
    \end{subfigure}
    \caption[\todo{short}]{
        CPU0 to CPU1 transfer bandwidth.
        (a) shows multi-threaded reads from CPU1 of an allocation on CPU0's NUMA Node.
        (b) shows multi-threaded writes from CPU0 to an allocation on CPU1's NUMA Node
    }
    \label{fig:cpu0-cpu1}
\end{figure}


\begin{algorithm}
    \caption{Algorithm to measure bandwidth achieved using \texttt{cudaMemcpy}}
    \label{alg:explicit-cpu-cpu}
    \begin{algorithmic}[1]
    \Statex
    \Function{Bandwidth}{$dst$, $src$, $transfer\_size$, $num\_iters$}
        \If{$src$ is GPU}
            \State \texttt{cudaSetDevice($src$}
        \Else \Comment{$src$ is CPU}
            \State \texttt{numa\_bind($src$}
        \EndIf
        \If{$dst$ is GPU}
        \State \texttt{cudaSetDevice($dst$)}
        \Else \Comment{$dst$ is CPU}
        \State \texttt{numa\_bind($dst$)}
        \EndIf

        \State $devPtr \gets$ \texttt{cudaMalloc($transfer\_size$)} \Comment{device allocation}
        \State $srcPtr \gets$ \texttt{malloc($transfer\_size$)} \Comment{or cudaHostAlloc()}

        \State $elapsed \gets infinity$ \Comment{minimum of $num\_iters$ observations}
        \For{$i \gets 1 \textrm{ to } num\_iters$}
            \State $start \gets$ walltime()
            \State \texttt{cudaMemcpy($dst$,$src$,$transfer\_size$)}
            \State \texttt{cudaDeviceSynchronize()}
            \State $end \gets$ walltime()
            \State $elapsed \gets$ min($elapsed$, $end-start$)
        \EndFor

        \Return $elapsed$
    \EndFunction

    \end{algorithmic}
\end{algorithm}



\section{CPU / GPU Transfers}
\label{sec:explicit-cpu-gpu}

Explicit CPU-GPU transfers are caused by the \texttt{cudaMemcpy} family of functions being invoked on one pointer to a host allocation and one pointer to a device allocation.
The host allocation may be pageable (created by \texttt{malloc} or \texttt{new}), or be created by \texttt{cudaMallocHost} or \texttt{cudaHostAlloc}.
The device allocation is created by \texttt{cudaMalloc}.
This section covers
\begin{itemize}
\item Comparing bandwidth achievable from pinned, pageable, and write-combining host allocations (Section~\ref{sec:explicit-pageable-pinned-wc})
\item The effect of device affinity on transfer performance (Section~\ref{sec:explicit-cpu-gpu-affinity})
\item Cases of observed anisotropic transfer properties (Section~\ref{sec:explicit-cpu-gpu-anisotropy})
\item Cases of differing performance on identical links (Section~\ref{sec:explicit-cpu-gpu-identical})
\end{itemize}

Algorithm~\ref{alg:explicit} is used to evaluate the achievable bandwidth when \texttt{cudaMemcpy} is used to transfer data between a GPU allocation and a pageable, pinned, or write-combining host allocation. 
The same algorithm can be used for these cases, because the same \texttt{cudaMemcpy} CUDA API call to transfer data can be used on a pointer pointing to any of the allocation types.
Depending on the source and destination types $src$ and $dst$, and the desired host allocation type, the corresponding CUDA or numactl APIs are called to bind later activities to the desired GPU or CPU.
Then, the CUDA and system allocators are invoked to produce $devPtr$ (a pointer to the device allocation) and $srcPtr$ (a pointer to the CPU allocation).
Then, \texttt{cudaMemcpy} is invoked $num\_iters$ times, and the fastest result is reported.
This helps remove any jitter from the results.

\begin{algorithm}
    \caption{Algorithm to measure bandwidth achieved using \texttt{cudaMemcpy}}
    \label{alg:explicit}
    \begin{algorithmic}[1]
    \Statex
    \Function{Bandwidth}{$dst$, $src$, $transfer\_size$, $num\_iters$}
        \If{$src$ is GPU}
            \State \texttt{cudaSetDevice($src$}
        \Else \Comment{$src$ is CPU}
            \State \texttt{numa\_bind($src$}
        \EndIf
        \If{$dst$ is GPU}
        \State \texttt{cudaSetDevice($dst$)}
        \Else \Comment{$dst$ is CPU}
        \State \texttt{numa\_bind($dst$)}
        \EndIf

        \State $devPtr \gets$ \texttt{cudaMalloc($transfer\_size$)} \Comment{device allocation}
        \State $srcPtr \gets$ \texttt{malloc($transfer\_size$)} \Comment{or cudaHostAlloc()}

        \State $elapsed \gets infinity$ \Comment{minimum of $num\_iters$ observations}
        \For{$i \gets 1 \textrm{ to } num\_iters$}
            \State $start \gets$ walltime()
            \State \texttt{cudaMemcpy($dst$,$src$,$transfer\_size$)}
            \State \texttt{cudaDeviceSynchronize()}
            \State $end \gets$ walltime()
            \State $elapsed \gets$ min($elapsed$, $end-start$)
        \EndFor

        \Return $elapsed$
    \EndFunction

    \end{algorithmic}
\end{algorithm}

\subsection{Comparison of Pageable, Pinned, and Write-Combining Host Allocations}
\label{sec:explicit-pageable-pinned-wc}

To contextualize other results presented in this chapter, Figure~\ref{fig:pageable-pinned-wc} shows the transfer performance from pageable, pinned, and write-combined allocations on CPU0 to device allocations on GPU0.
These performance curves exhibit features common throughout this chapter:
\begin{itemize}
\item For small transfer sizes, the time is dominated by overhead introduced by the abstraction layer.
\item For large transfer sizes, the time is dominated by bandwidth limits on the exercised physical link.
\item The performance may vary smoothly across intermediate transfer sizes, or exhibit more complicated behavior.
\end{itemize}

For transfers from pageable host allocations to GPU0, (Figure~\ref{fig:pageable-cpu0-gpu0}), the achievable bandwidth can be substantially lower than the advertised hardware specifications suggest.
The achieved 30 GB/s transfer bandwidth on S822LC is 75\% of the theoretical 40 GB/s transfer provided by the NVLink 1.0 interconnect between CPU0 and GPU0 (Section~\ref{sec:minsky}).
On AC922, howoever, the 30 GB/s achieved bandwidth is substantially lower than the 75 GB/s advertised by the NVLink 2.0.
Furthermore, the topologically-similar S822LC and AC922 machines do not have similar shaped curves.
On S822LC and AC922, write-combined host allocations do not affect transfer performance.

\begin{figure}[ht]
    \centering
    \begin{subfigure}[b]{0.3\textwidth}
        \includegraphics[width=\textwidth]{figures/generated/pageable_cpu0-gpu0.pdf}
        \caption{}
        \label{fig:pageable-cpu0-gpu0}
    \end{subfigure}
    ~
    \begin{subfigure}[b]{0.3\textwidth}
        \includegraphics[width=\textwidth]{figures/generated/pinned_cpu0-gpu0.pdf}
        \caption{}
        \label{fig:}
    \end{subfigure}
    ~
    \begin{subfigure}[b]{0.3\textwidth}
        \includegraphics[width=\textwidth]{figures/generated/wc_cpu0-gpu0.pdf}
        \caption{}
        \label{fig:}
    \end{subfigure}
    \caption[\todo{short}]{
        \texttt{cudaMemcpy} performance for CPU0 to GPU0 transfers from 
        (a) pageable allocations,
        (b) pinned allocations, and
        (c) write-combining allocations.
        For pageable transfers, bandwidth may be much lower 
    }
    \label{fig:pageable-pinned-wc}
\end{figure}

\subsection{Affinity}
\label{sec:explicit-cpu-gpu-affinity}

There is a distinct performance difference due to affinity, or transfers between components within a triad (CPU0-GPU0-GPU1 or CPU1-GPU2-GPU3) and components across triads.
Figure~\ref{fig:cpu-gpu-affinity-direction} shows some cases where the device physical affinity affects the performance of the logical communication path.
Table~\ref{tab:cpu-gpu-affinity} summarizes the effects.

\begin{figure}[ht]
    \centering
    \begin{subfigure}[b]{0.45\textwidth}
        \includegraphics[width=\textwidth]{figures/generated/minsky_pageable_affinity.pdf}
        \caption{}
        \label{fig:minsky_pageable_affinity}
    \end{subfigure}
    ~
    \begin{subfigure}[b]{0.45\textwidth}
        \includegraphics[width=\textwidth]{figures/generated/hal_pageable_affinity.pdf}
        \caption{}
        \label{fig:hal_pageable_affinity}
    \end{subfigure}
    \\
    \begin{subfigure}[b]{0.45\textwidth}
        \includegraphics[width=\textwidth]{figures/generated/minsky_pinned_affinity_cpu0.pdf}
        \caption{}
        \label{fig:minsky_pinned_affinity}
    \end{subfigure}
    ~
    \begin{subfigure}[b]{0.45\textwidth}
        \includegraphics[width=\textwidth]{figures/generated/hal_pinned_affinity.pdf}
        \caption{}
        \label{fig:hal_pinned_affinity}
    \end{subfigure}
    \caption[\todo{short}]{
    Example effect of affinity-affected and anisotropic logical communication bandwidth
    (a) shows transfers involving pageable allocations on CPU0, to device allocations on GPU0 and GPU2 in both directions on S822LC.
    The transfers are highly anisotropic, with GPU-to-CPU transfers being nearly 50\% of the speed.
    Affinity also has a small effect of around 5\% for CPU-to-GPU transfers and 20\% for GPU-to-CPU transfers.
    (b) is the same scenario on AC922.
    Again, transfers are highly anisotropic. GPU-to-CPU communication is also strongly affected by affinity.
    (c) and (d) are the same as (a) and (b), except with pinned transfers.
    On S822LC, bandwidth is affected both by affinity and direction, though the effect is not as strong as with pageable stransfers.
    For AC922, affinity has a much stronger effect, though anisotropy is still present for remote transfers.
    }
    \label{fig:cpu-gpu-affinity-direction}
\end{figure}

\begin{table}[ht]
    \centering
    \caption[Affinity and Logical Communication Bandwidth]{Effect of device affinity on logical communication bandwidth}
    \label{tab:cpu-gpu-affinity}
    \begin{tabular}{|c|c|c|c|}
    \hline
    \textbf{Transfer Kind}     & \textbf{S822LC}                                      & \textbf{AC922} & \textbf{DGX-1} \\ \hline 
    Pageable $\rightarrow$ GPU & \checkmark (Fig.~\ref{fig:minsky_pageable_affinity}) & $\times$   (Fig.~\ref{fig:hal_pageable_affinity}) & \\ \hline
    Pageable $\leftarrow$ GPU  & \checkmark (Fig.~\ref{fig:minsky_pageable_affinity}) & \checkmark (Fig.~\ref{fig:hal_pageable_affinity}) & \\ \hline
    Pinned $\rightarrow$ GPU   & \checkmark (Fig.~\ref{fig:minsky_pinned_affinity})   & \checkmark (Fig.~\ref{fig:hal_pinned_affinity})   & \\ \hline
    Pinned $\leftarrow$ GPU    & \checkmark (Fig.~\ref{fig:minsky_pinned_affinity})   & \checkmark (Fig.~\ref{fig:hal_pinned_affinity})   & \\ \hline
    \end{tabular}
\end{table}

\subsection{Anisotropy}
\label{sec:explicit-cpu-gpu-anisotropy}

Figure~\ref{fig:cpu-gpu-affinity-direction} also shows some cases where the direction of a logical transfer affects the performance.
Table~\ref{tab:explicit-anisotropy} summarizes the effects.

\begin{table}[ht]
    \centering
    \caption[Summary of Host-Device Transfer Anisotropy]{Summary of Host-Device Transfer Anisotropy}
    \label{tab:explicit-anisotropy}
    \begin{tabular}{|c|c|c|c|}
    \hline
    \textbf{Transfer Kind}                         & \textbf{S822LC}     & \textbf{AC922} & \textbf{DGX-1} \\ \hline 
    Pageable $\leftrightarrow$ GPU (local)         & \checkmark (Fig.~\ref{fig:minsky_pageable_affinity}) & \checkmark (Fig.~\ref{fig:hal_pageable_affinity}) & \\ \hline
    Pageable $\leftrightarrow$ GPU (remote)        & \checkmark (Fig.~\ref{fig:minsky_pageable_affinity}) & \checkmark (Fig.~\ref{fig:hal_pageable_affinity}) & \\ \hline
    Pinned $\leftrightarrow$ GPU (local)           & \checkmark (Fig.~\ref{fig:minsky_pinned_affinity})   & for intermediate sizes (Fig.~\ref{fig:hal_pinned_affinity}) & \\ \hline
    Pinned $\leftrightarrow$ GPU (remote)          & \checkmark (Fig.~\ref{fig:minsky_pinned_affinity})   & \checkmark             (Fig.~\ref{fig:hal_pinned_affinity}) & \\ \hline
    \end{tabular}
\end{table}

\subsection{Differences between Identical Transfers}
\label{sec:explicit-cpu-gpu-identical}

Figure~\ref{fig:minsky_pageable_cpu1-gpu01} shows transfer performance on two different identical links: CPU0-GPU0 and CPU0-GPU1.
Table~\ref{tab:explicit-identical} summarizes scenarios where the transfer performance differs on identical links.

\begin{figure}[ht]
    \centering

    \begin{subfigure}[b]{0.45\textwidth}
        \includegraphics[width=\textwidth]{figures/generated/minsky_pageable_cpu1-gpu01.pdf}
        \caption{}
        \label{fig:}
    \end{subfigure}
    ~
    \begin{subfigure}[b]{0.45\textwidth}
        \includegraphics[width=\textwidth]{figures/generated/minsky_pageable_gpu02-cpu01.pdf}
        \caption{}
        \label{fig:}
    \end{subfigure}

    \caption[\todo{short}]{
        (a) shows transfer bandwidth from a pageable allocation on CPU1 to an allocation on GPU0 or GPU1.
        (b) shows transfer bandwidth from an allocation on GPU0 or GPU2 to a pageable allocation CPU0 or CPU1.
        Both of these scenarios are identical from a logical and hardware perspective, yet performance varies substantially.}
    \label{fig:minsky_pageable_cpu1-gpu01}
\end{figure}

\begin{table}[ht]
    \centering
    \caption[Matrix: Transfer rate vary on identical links]{Scenarios where bandwidth is observed to differ on identical links}
    \label{tab:explicit-identical}
    \begin{tabular}{|c|c|c|c|}
    \hline
    \textbf{Transfer Kind}     & \textbf{S822LC}     & \textbf{AC922} & \textbf{DGX-1} \\ \hline 
    Pageable $\rightarrow$ GPU & \checkmark (Fig.~\ref{fig:minsky_pageable_cpu1-gpu01}) & $\times$ & \\ \hline
    Pageable $\leftarrow$ GPU  & \checkmark (Fig.~\ref{fig:minsky_pageable_cpu1-gpu01}) & $\times$ & \\ \hline
    Pinned $\rightarrow$ GPU   & $\times$                                                   & $\times$ & \\ \hline
    Pinned $\leftarrow$ GPU    & $\times$                                                   & $\times$ & \\ \hline
    \end{tabular}
\end{table}

\section{GPU / GPU Transfers}
\label{sec:explicit-gpu-gpu}

Explicit GPU-GPU transfers are caused by the \texttt{cudaMemcpy} family of functions being invoked on pointers to device allocations created with \texttt{cudaMalloc}.
Unlinke the different types of host allocations CPU-GPU transfers described in Section~\ref{sec:explicit-cpu-gpu}, this section only refers to a single kind of device allocation.
Device allocations come with the concept of peer access, discussed in Section~\ref{sec:cuda-peer}.
This section covers
\begin{itemize}
\item The effect of peer access on transfer bandwidth (Section~\ref{sec:explicit-peer-bandwidth})
\item Cases of observed anisotropic transfer properties (Section~\ref{sec:explicit-peer-direction})
\item Cases of differing performance on identical links (Section~\ref{sec:explicit-peer-identical})
\end{itemize}

Algorithm~\ref{alg:explicit} is used to evaluate the achievable GPU-GPU transfer bandwidth.
First, peer access is enabled or disabled depending on the experimental configuration.
Then, \texttt{cudaMalloc} is used to create allocations of $transfer\_size$ bytes pointed to by $srcPtr$ and $dstPtr$.
Finally, the achievable bandwidth is measured using the wall time for $num\_iters$ iterations, and the minimum elapsed time is reported, to help remove jitter from the results.

\begin{algorithm}
    \caption{Measuring explicit \texttt{cudaMemcpy} performance}
    \label{alg:explicit}
    \begin{algorithmic}[1]
    \Statex
    \Function{Bandwidth}{$dst$, $src$, $transfer\_size$, $num\_iters$, $peer\_access$}
        \If{$peer\_access$}
            \State \texttt{cudaSetDevice($src$)}
            \State \texttt{cudaDeviceEnablePeerAccess($dst$)}
            \State \texttt{cudaSetDevice($dst$)}
            \State \texttt{cudaDeviceEnablePeerAccess($src$)}
        \Else
            \State \texttt{cudaSetDevice($src$)}
            \State \texttt{cudaDeviceDisablePeerAccess($dst$)}
            \State \texttt{cudaSetDevice($dst$)}
            \State \texttt{cudaDeviceDisablePeerAccess($src$)}        
        \EndIf

        \State \texttt{cudaSetDevice($src$)} \Comment{Source allocation}
        \State $srcPtr \gets$ \texttt{cudaMalloc($transfer\_size$)}

        \State \texttt{cudaSetDevice($dst$)} \Comment{Destination allocation}
        \State $dstPtr \gets$ \texttt{cudaMalloc($transfer\_size$)}

        \State $elapsed \gets infinity$ \Comment{minimum of $num\_iters$ observations}
        \For{$i \gets 1 \textrm{ to } num\_iters$}
            \State $start \gets$ walltime()
            \State \texttt{cudaMemcpy($dst$,$src$,$transfer\_size$)}
            \State $end \gets$ walltime()
            \State $elapsed \gets$ min($elapsed$, $end-start$)
        \EndFor

    \Return $elapsed$
    \EndFunction

    \end{algorithmic}
\end{algorithm}

\subsection{Transfer Rate and Peer Access}
\label{sec:explicit-peer-bandwidth}

Figure~\ref{fig:explicit-peer} shows the performance of a variety of GPU-GPU transfers with and without peer access.
Peer access has a large effect on the bandwidth of local GPU-GPU transfers.
In some cases, when peer access is disabled, local access performance varies across identical links.
Enabling peer access can also reduce the bandwidth of remote accesses.
Table~\ref{tab:explicit-peer-rate} summarizes the scenarios where peer access affects transfer performance.

\begin{figure}[ht]
    \centering
    \begin{subfigure}[b]{0.3\textwidth}
        \includegraphics[width=\textwidth]{figures/generated/minsky_memcpy_local.pdf}
        \caption{}
        \label{fig:}
    \end{subfigure}
    ~
    \begin{subfigure}[b]{0.3\textwidth}
        \includegraphics[width=\textwidth]{figures/generated/hal_peer_local.pdf}
        \caption{}
        \label{fig:explicit-hal-peer-local}
    \end{subfigure}
    ~
    \begin{subfigure}[b]{0.3\textwidth}
        \includegraphics[width=\textwidth]{figures/generated/hal_peer_remote.pdf}
        \caption{}
        \label{fig:}
    \end{subfigure}
    \caption[\todo{short}]{
        The effect of enabling peer access
        (a) on S822LC for local GPU-GPU transfers,
        (b) on AC922 for local GPU-GPU transfers, and
        (c) on AC922 for remote GPU-GPU transfers.
        S822LC does not support peer access for remote GPU-GPU transfers.
    }
    \label{fig:explicit-peer}
\end{figure}


\begin{table}[ht]
    \centering
    \caption[Matrix: Transfer rate affected by peer access]{Is transfer rate affected by peer access?}
    \label{tab:explicit-peer-rate}
    \begin{tabular}{|c|c|c|c|}
    \hline
    \textbf{Transfer Kind}       & \textbf{S822LC} & \textbf{AC922} & \textbf{DGX-1} \\ \hline 
    GPU $\rightarrow$ Local GPU  & \checkmark      & \checkmark     & \\ \hline
    GPU $\rightarrow$ Remote GPU & N/A             & \checkmark     & \\ \hline
    \end{tabular}
\end{table}


\subsection{Transfer Anisotropy with Peer Access Disabled}
\label{sec:explicit-peer-direction}

Anisotropy in GPU-GPU transfers is also observed on S822LC and AC922.
This anisotropy is not consistent with anisotpropy observed on the intervening links.
For example, consider Figure~\ref{fig:explicit-peer-anisotropy}a-c.
Figure~\ref{fig:minsky-explicit-nopeer-remote} shows anisotropy along the remote GPU0-GPU2 transfer.
Figures~\ref{fig:minsky-explicit-path-gpu0-gpu2}~and~\ref{fig:minsky-explicit-path-gpu2-gpu0} show pinned transfer speeds along GPU0-CPU0-CPU1-GPU2 and GPU2-CPU1-CPU0-GPU0 paths.
The observed GPU0-GPU2 bandwidth is sometimes higher than the higher observed bandwidth on the path components, and sometimes lower than the lowest path component.
Figure~\ref{fig:explicit-peer-anisotropy} highlights anisotropic remote GPU-GPU transfers on S822LC and AC922.

\begin{figure}[ht]
    \centering
    \begin{subfigure}[b]{0.3\textwidth}
        \includegraphics[width=\textwidth]{figures/generated/minsky_nopeer_remote.pdf}
        \caption{}
        \label{fig:minsky-explicit-nopeer-remote}
    \end{subfigure}
    ~
    \begin{subfigure}[b]{0.3\textwidth}
        \includegraphics[width=\textwidth]{figures/generated/minsky_path_gpu0-gpu2.pdf}
        \caption{}
        \label{fig:minsky-explicit-path-gpu0-gpu2}
    \end{subfigure}
    ~
    \begin{subfigure}[b]{0.3\textwidth}
        \includegraphics[width=\textwidth]{figures/generated/minsky_path_gpu2-gpu0.pdf}
        \caption{}
        \label{fig:minsky-explicit-path-gpu2-gpu0}
    \end{subfigure}
    \\
    \begin{subfigure}[b]{0.3\textwidth}
        \includegraphics[width=\textwidth]{figures/generated/hal_nopeer_remote.pdf}
        \caption{}
        \label{fig:}
    \end{subfigure}
    \caption[\todo{short}]{
        Examples of GPU-GPU transfer anisotropy that present when peer access is disabled.
        (a) shows anisotropy on transfers between GPU0 and GPU2.
        With peer access disabled, that transfer passes along the path from GPU0 to CPU0, CPU0 to CPU1, and CPU1 to GPU2 (both directions shown in (b) and (c)).
        Although (b) and (c) are the same, the anisotropy still exists in the aggregated logical link.
        (d) shows anisotropy on AC922.
    }
    \label{fig:explicit-peer-anisotropy}
\end{figure}

\begin{table}[ht]
    \centering
    \caption[GPU-GPU Transfer Anisotropy]{GPU-GPU Transfer Anisotropy}
    \label{tab:explicit-peer-direction}
    \begin{tabular}{|c|c|c|c|}
    \hline
    \textbf{Transfer Kind}                           & \textbf{S822LC} & \textbf{AC922} & \textbf{DGX-1} \\ \hline 
    GPU $\leftrightarrow$ Local GPU  (peer enabled)  & $\times$        & $\times$       & \\ \hline
    GPU $\leftrightarrow$ Remote GPU (peer enabled)  & N/A             & $\times$       & \\ \hline
    GPU $\leftrightarrow$ Local GPU  (peer disabled) & $\times$        & $\times$       & \\ \hline
    GPU $\leftrightarrow$ Remote GPU (peer disabled) & \checkmark      & \checkmark     & \\ \hline
    \end{tabular}
\end{table}

\subsection{Transfer Rate on Identical Transfers}
\label{sec:explicit-peer-identical}

Like CPU-GPU transfers, different performance is observed on identical GPU-GPU transfers when peer access is disabled.
Figure~\ref{fig:explicit-nopeer-identical} show some example scenarios.
Table~\ref{tab:explicit-identical} summarizes the cases where differing performance is observed.

\begin{figure}[ht]
    \centering
    \begin{subfigure}[b]{0.4\textwidth}
        \includegraphics[width=\textwidth]{figures/generated/hal_nopeer_identical-local.pdf}
        \caption{}
        \label{fig:}
    \end{subfigure}
    ~
    \begin{subfigure}[b]{0.4\textwidth}
        \includegraphics[width=\textwidth]{figures/generated/hal_nopeer_identical-remote.pdf}
        \caption{}
        \label{fig:}
    \end{subfigure}
    \\
    \begin{subfigure}[b]{0.4\textwidth}
        \includegraphics[width=\textwidth]{figures/generated/minsky_nopeer_identical-local.pdf}
        \caption{}
        \label{fig:}
    \end{subfigure}
    ~
    \begin{subfigure}[b]{0.4\textwidth}
        \includegraphics[width=\textwidth]{figures/generated/minsky_nopeer_identical-remote.pdf}
        \caption{}
        \label{fig:}
    \end{subfigure}
    \caption[\todo{short}]{
        Scenarios where identical logical transfers show different performance when peer access is disabled.
        (a) and (b) show the bevahior on AC922 for local and remote transfers.
        (c) and (d) show the same for S822LC.
    }
    \label{fig:explicit-nopeer-identical}
\end{figure}

\begin{table}[ht]
    \centering
    \caption[Matrix: Transfer rate on Identical Links]{Does the GPU transfer rate vary on identical links?}
    \label{tab:explicit}
    \begin{tabular}{|c|c|c|c|}
    \hline
    \textbf{Transfer Kind}                           & \textbf{S822LC} & \textbf{AC922} & \textbf{DGX-1} \\ \hline 
    GPU $\leftrightarrow$ Local GPU  (peer enabled)  & $\times$        & $\times$       & \\ \hline
    GPU $\leftrightarrow$ Remote GPU (peer enabled)  & N/A             & $\times$       & \\ \hline
    GPU $\leftrightarrow$ Local GPU  (peer disabled) & \checkmark      & \checkmark     & \\ \hline
    GPU $\leftrightarrow$ Remote GPU (peer disabled) & \checkmark      & \checkmark     & \\ \hline
    \end{tabular}
\end{table}

