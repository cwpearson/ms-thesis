\chapter{Application Characterization}


Directly analogous to the hardware model, this work models applications as a collection of communicating data sources and sinks (endpoints).
This facilitates the performance modeling described in Chapter~\ref{ch:performance}.
Through it is possible to generate such application models by hand, in general, this approach is not feasible for complicated applications.
Furthermore, applications are frequently updated, which could require a new model.
This work proposes an approach for automatically generating these application models.

\section{Application Model}

The goal of the application record is to track how an application produces, moves, and consumes data.
To that end, the application is broken down into data sources and sinks.
Memory allocations, file creation/access, network activity, and computation kernels all represent possible producers or consumers of data.

Whenever data is moved, it moves from a source to a sink.
Table~\ref{tab:source-sink-example} gives some examples of data movement and the corresponding sources and sinks.

\begin{table}[h]
    \centering
    \caption{\todo{caption}}
    \label{tab:source-sink-example}
    \begin{tabular}{|c|c|c|}
    \hline
    \textbf{Application Activity} & \textbf{Data Source} & \textbf{Data Sink} \\ \hline
    File system read  & file & calling function \\ \hline
    File system write  & file & calling function \\ \hline
    \textbf{cudaMemcpy()} & allocation & allocation \\ \hline
    function call & pointer arguments & pointer arguments \\ \hline
    \end{tabular}
\end{table}

The application can therefore be represented by a graph $G_a = \{E_a,V_a\}$ where $E_a$ is a set of edges representing data transfer, and $V_a$ is a set of vertices representing data endpoints.

\section{Application Monitoring}


A custom application trace is generated during an application execution to build the application graph.
\todo{apptracer} is built on top of CUDA Profiling Tools Interface~\cite{nvidia2017cupti}(CUPTI) and the Linux \texttt{LD\_PRELOAD}~\cite{kerrisk2017ld} mechanism.
CUPTI(Section~\ref{sec:cupti}) allows \todo{apptracer} to provide a callback function that is invoked at every CUDA runtime or driver call, and also allows \todo{apptracer} to collect any performance metrics the GPU exposes.
The callback function records relevant information, including the wall time when the CUDA runtime function is invoked, its arguments, and the device and stream associated with the call.
In this way, detailed information about data transfers from runtime functions can be collected.


Additionally, conservative data consumption and generation from CUDA kernels can be inferred.
Any arguments that point to previously-recorded allocations are assumed to modify that allocation.
\todo{through monitoring the device metrics, it is possible to understand how much data is read and written through those pointers in aggregate.}



\subsection{Communication Link Traffic}

\outline{PCI traffic}

\outline{NVLink traffic}

\section{Application Modeling Case Studies}
\subsection{Chai}
\subsection{Caffe}
\subsection{Graph Challenge}
\subsection{Lulesh}