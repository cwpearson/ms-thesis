\chapter{Conclusion}
\label{ch:conclusion}

The conclusions drawn from the study are given in the last chapter.
The last chapter also can include discussions of the advantages and limitations of the results obtained, comparisons with previous work, possible applications for the results, and suggestions for future work.

\outline{

Possible future work:

    Library interface for moving data

    Characterizing direct peer access
    Monitoring application direct peer access
    More communication links {ethernet}
    Proper characterization of multiple paths when available
    Multi-node system model
    Proper SMP bus topology  detection
    CUDA system atomics characterization?
    non-compressible data for cuda memcpy?

    Investigating the hardware topology scheduling problem
    Simulation and Replay of Profiles
}

\todo{Investigating performance details (radare2, callgrind)?}

Hardware level measurement, and software measurement, and connect base on measurements.

For thesis, mention that the C path could involve multiple hops in A, and this could be measured, we take this as a-priori.

More interesting to use some example, b/w pascals in this p8 system, this is what we measured.

Sprinkle some plot results in to the background examples

contention on hardware links

effect of contention at any level

possible to actually determine SMP bus topology?


extend hardware enumeration to linux network and block devices
